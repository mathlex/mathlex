\documentclass{article}

\usepackage[top=1in, bottom=1in, left=0.75in, right=0.75in]{geometry}
\usepackage[T1]{fontenc}
\usepackage{amsmath}
\usepackage{amssymb}
\usepackage{url}

\setlength{\parindent}{0em}
\setlength{\parskip}{1em}

\begin{document}
  \title{List of Symbols\thanks{Supported in part by NSF DUE CCLI/TUES Grants 0737248 / 1123255.}\\
  AMS Regional Meeting, Lawrence, KS}
  \author{Matthew J. Barry  <komputerwiz.matt@gmail.com> \\
          Philip B. Yasskin <yasskin@math.tamu.edu>\\
          Department of Mathematics, Texas A\&M University}
  \date{March 31, 2012}
  \maketitle
  
  \vspace{-15 pt}
  
  Matthew Barry is writing an HTML5/javascript/MathJax parser/renderer as an Honors Senior Thesis. It may be viewed at: \qquad \url{http://custom.komputerwiz.net:8000/~matthew/interpreter-2/} \qquad
  Below is a list of the symbols that can be entered, organized syntax usage and by mathematical topic. Please send comments and suggestions for additions and changes to the authors.
  
  \vspace{-10 pt}

  \section{Symbols by Syntax Usage} % (fold)
  \label{sec:syntax}
  \subsection{Constants} % (fold)
  \label{sub:constants}
  \begin{table}[!h]
    \centering
    \begin{tabular}{|c|c|c|l|}
      \hline
      \textbf{Name} & \textbf{Symbol} & \textbf{Code} & \textbf{Description} \\
      \hline\hline
      Pi & \( \pi \) & \texttt{\#pi, \#p} & 3.14\(\cdots\)\\
      E & \( \mathrm{e} \) & \texttt{\#e} & 2.718\(\cdots\), Natural Base, Euler-Napier number \\
      gamma & \( \gamma \) & \texttt{\#gamma} & 0.577\(\cdots\), Euler-Mascheroni constant\\
      Imaginary Unit & \( i \) & \texttt{\#i} & \( \sqrt{-1} \) \\
      Infinity & \( \infty \) & \texttt{\#infinity}, \texttt{infinity} & \\
      \hline
      True & \( \mathbf{T} \) & \texttt{\#T}, \texttt{\#t}, \texttt{\#true}, \texttt{true} & \\
      False & \( \mathbf{F} \) & \texttt{\#F}, \texttt{\#f}, \texttt{\#false}, \texttt{false} & \\
      \hline
      Natural Numbers & \( \mathbb{N} \) & \texttt{\#N} & \\
      Integer Ring & \( \mathbb{Z} \) & \texttt{\#Z} & \\
      Rational Field & \( \mathbb{Q} \) & \texttt{\#Q} & \\
      Real Field & \( \mathbb{R} \) & \texttt{\#R} & \\
      Complex Field & \( \mathbb{C} \) & \texttt{\#C} & \\
      Quaternion Ring & \( \mathbb{H} \) & \texttt{\#H} & Hamilton numbers\\
      Octonion Algebra & \( \mathbb{O} \) & \texttt{\#O} & Cayley numbers, Type ``Oh''.\\
      \hline
      Universal Set & \( \mathbb{U} \) & \texttt{\#U} & \\
      Empty Set & \( \emptyset \) & \texttt{\#empty}, \texttt{\{\}} & \\
      \hline
      Zero Vector & \( \vec0 \) & \texttt{\#v0} & \\
      \( x \) Unit Vector & \( \hat\imath \) & \texttt{\#vi} & \\
      \( y \) Unit Vector & \( \hat\jmath \) & \texttt{\#vj} & \\
      \( z \) Unit Vector & \( \hat{k} \) & \texttt{\#vk} & \\
      \hline
      Zero Matrix & \( \mathbb{O} \) & \texttt{\#0} & Type ``zero''.\\
      Unit Matrix & \( \mathbb{I} \) & \texttt{\#1} & Type ``one''.\\
      \hline
    \end{tabular}
  \end{table}
  % subsection constants (end)

  \newpage

  \subsection{Unary Operators} % (fold)
  \label{sub:unary_ops}
  \begin{table}[!h]
    \centering
    \begin{tabular}{|c|c|c|l|}
      \hline
      \textbf{Name} & \textbf{Symbol} & \textbf{Code} & \textbf{Description} \\
      \hline\hline

      Positive & \( +a \) & \texttt{+a} & \\
      Negative & \( -a \) & \texttt{-a} & \\
      Positive/Negative & \( \pm a \) & \texttt{\&pm a, +/-a} & \\
      Negative/Positive & \( \mp a \) & \texttt{\&mp a, -/+a} & \\
      Square Root & \( \sqrt{a} \) & \texttt{sqrt(a)} & \\
      Factorial & \( n! \) & \texttt{n!} & \\
      Natural Exponential & \( \exp(a) \) & \texttt{exp(a)} & \\
      Natural Logarithm & \( \ln(a) \) & \texttt{ln(a)} & \\
      Logical Negation & \( \neg p \) & \texttt{not p, \~{}p, !p} & \\
      \hline
      Prime derivative & \( f' \) & \texttt{f'} & Derivative w.r.t. x or first or only variable\\
      Dot derivative & \( \dot{f} \) & \texttt{f.} & Derivative w.r.t. t or second variable\\
      Differential & \( \mathrm{d}x \) & \texttt{\&d x} & \\
      Partial Differential & \( \partial x \) & \texttt{\&pd x} & \\
      Vector & \( \vec{u} \) & \texttt{\&v u} & \\
      Unit Vector & \( \hat{u} \) & \texttt{\&u u} & \\
      \hline
    \end{tabular}
  \end{table}
  % subsection unary_ops (end)

  \subsection{Binary Operators} % (fold)
  \label{sub:binary_ops}
  \begin{table}[!h]
    \centering
    \begin{tabular}{|c|c|c|l|}
      \hline
      \textbf{Name} & \textbf{Symbol} & \textbf{Code} & \textbf{Description} \\
      \hline\hline

      Plus & \( a + b \) & \texttt{a+b} & Addition \\
      Minus & \( a - b \) & \texttt{a-b} & Subtraction \\
      Plus/Minus & \( a \pm b \) & \texttt{a \&pm b, a+/-b} & \\
      Minus/Plus & \( a \mp b \) & \texttt{a \&mp b, a-/+b} & \\
      Times & \( a \cdot b \) & \texttt{a*b} & Multiplication \\
      Divide & \( \dfrac{a}{b} \) & \texttt{a/b} & Division \\
      Power & \( a^b \) & \texttt{a\^{}b}, \texttt{a**b} & Exponentiation \\
      \( n \)-th Root & \( \sqrt[n]{a} \) & \texttt{root(a, n)} & \\
      Logarithm with Base & \( \log_b{a} \) & \texttt{log(a, b)} & \\
      Modulus & \( a \pmod{n} \) & \texttt{a\%n}, \texttt{a mod n} & \\
      \hline
      Subscript & \( a_b \) & \texttt{a \&\_ b} & Indexing \\
      Superscript & \( a^b \) & \texttt{a \&\^{} b} & Indexing \\
      \hline
      Set Union & \( a \cup b \) & \texttt{a union b} & \\
      Set Intersection & \( a \cap b \) & \texttt{a intersect b} & \\
      Set Difference & \( a \setminus b \) & \texttt{a \textbackslash \ b} & \\
      Such That & \( p : q \), \( p ~|~ q \) & \texttt{p:q, p|q} & Used with set builder and quantifiers \\
      \hline
      Conjunction & \( p \wedge q \) & \texttt{p and q}, \texttt{p \&\& q} & Logical AND \\
      Disjunction & \( p \vee q \) & \texttt{p or q}, \texttt{p || q} & Logical OR \\
      Exclusion & \( p \oplus q \) & \texttt{p xor q} & Logical XOR \\
      Conditional & \( p \rightarrow q \) & 
        \begin{tabular}{c}
          \texttt{p implies q}, \texttt{p -> q}, \texttt{q if p}, \\
          \texttt{q when p}, \texttt{q whenever p}
        \end{tabular} & \\
      Biconditional & \( p \leftrightarrow q \) & \texttt{p iff q}, \texttt{p <-> q} & \\
      \hline
      Function Composition & \( f \circ g \) & \texttt{f @ g} & \\
      \hline
      Dot Product & \( \vec{a} \cdot \vec{b} \) & \texttt{\&v a \&. \&v b} & \\
      Cross Product & \( \vec{a} \times \vec{b} \) & \texttt{\&v a \&x \&v b} & \\
      \hline
    \end{tabular}
  \end{table}
  % subsection binary_ops (end)

  \newpage

  \subsection{Relations} % (fold)
  \label{sub:relations}
  \begin{table}[!h]
    \centering
    \begin{tabular}{|c|c|c|l|}
      \hline
      \textbf{Name} & \textbf{Symbol} & \textbf{Code} & \textbf{Description} \\
      \hline\hline

      Equal & \( = \) & \texttt{=}, \texttt{==} & \\
      Not Equal & \( \ne \) & \texttt{!=}, \texttt{/=}, \texttt{<>} & \\
      Less than & \( < \) & \texttt{<} & \\
      Greater than & \( > \) & \texttt{>} & \\
      Less than or Equal & \( \le \) & \texttt{<=} & \\
      Greater than or Equal & \( \ge \) & \texttt{>=} & \\
      \hline
      Subset & \( \subseteq \) & \texttt{subset} & \\
      Superset & \( \supseteq \) & \texttt{superset, supset} & \\
      Proper Subset & \( \subset \) &
        \texttt{propersubset, propsubset, psubset} & \\
      Proper Superset & \( \supset \) &
        \begin{tabular}{c}
          \texttt{propersuperset, propsuperset, psuperset} \\
          \texttt{propersupset, propsupset, psupset}
        \end{tabular} &  \\
      Inclusion & \( \in \) & \texttt{in} & \\
      \hline
      Equivalent & \( \equiv \) & \texttt{===}, \texttt{equiv} & \\
      Not Equivalent & \( \not\equiv \) & \texttt{!==}, \texttt{/==}, \texttt{nequiv} & \\
      \hline
    \end{tabular}
  \end{table}
  % subsection relations (end)

  \subsection{Delimiters} % (fold)
  \label{sub:delimiters}
  \begin{table}[!h]
    \centering
    \begin{tabular}{|c|c|c|l|}
      \hline
      \textbf{Name} & \textbf{Symbol} & \textbf{Code} & \textbf{Description} \\
      \hline\hline
      Parentheses & \( \left( \ \right) \) & \texttt{( )} & Order of operation \\
      Square Brackets & \( \left[ \ \right] \) & \texttt{[ ]} & Lists \\
      Curly Braces & \( \left\{ \ \right\} \) & \texttt{\{ \}} & Sets \\
      Angle Brackets & \( \left\langle \ \right\rangle \) & \texttt{< >}, \texttt{<: :>} & Vectors \\
      Vertical Bars & \( \left| \ \right| \) & \texttt{| |}, \texttt{|: :|} & Absolute Value, Length, Determinant, Norm \\
      Double Bars & \( \left\| \ \right\| \) & \texttt{|| ||}, \texttt{||: :||} & Length, Norm \\
      \hline
      Subscript & \( a_b \) & \texttt{a \&\_ b} & Indexing \\
      Superscript & \( a^b \) & \texttt{a \&\^{} b} & Indexing \\
      Such That & \( p : q \), \( p ~|~ q \) & \texttt{p : \!\!q}, \texttt{p | q} & Used with set builder and quantifiers \\
      \hline
      Open Interval & \( \left( a,b \right) \) & \texttt{(:a,b:)} & Exclusive Range Delimiters \\
      Closed Interval & \( \left[ a,b \right] \) & \texttt{[:a,b:]} & Inclusive Range Delimiters \\
      Half-Open Interval & \( \left[ a,b \right) \) & \texttt{[:a,b:)} & Mixed Range Delimiters \\
      \hline
    \end{tabular}
  \end{table}
  Note that some delimiters have more than one format either with or without colons. Namely, absolute value can be
  written as \texttt{| ... |} or \texttt{|: ... :|}, norm can be written as \texttt{|| ... ||} or \texttt{||: ... :||},
  and vector literals can be surrounded by either \texttt{< ... >} or \texttt{<: ... :>}. Of the listed alternate
  delimiters, those without colons are \emph{context-aware} in that they have different meanings and therefore cannot be automatically
  matched by the Lexer. Additionally, if an expression opened with one type of delimiter, it must be closed with the
  same type (i.e. context-aware vs. specialized).
  % subsection delimiters (end)

  \newpage

  \subsection{Functions} % (fold)
  \label{sub:functions}
  \begin{table}[!h]
    \centering
    \begin{tabular}{|c|c|c|l|}
      \hline
      \textbf{Name} & \textbf{Symbol} & \textbf{Code} & \textbf{Description} \\
      \hline\hline

      Trig & \( \sin(\theta) \), \ldots & \texttt{sin(theta)}, \ldots & Also cos, tan, cot, sec, csc \\
      Inverse Trig & \( \arcsin(x) \), \ldots & \texttt{arcsin(x)}, \ldots & Also arccos, arctan, arccot, arcsec, arccsc \\
      Hyperbolic Trig & \( \sinh(\lambda) \), \ldots & \texttt{sinh(lambda)}, \ldots & Also cosh, tanh, coth, sech, csch \\
      Inv. Hyp. Trig & \( \mathrm{arcsinh}(x) \), \ldots & \texttt{arcsinh(x)}, \ldots & Also arccosh, arctanh, arccoth, arcsech, arccsch \\
      \hline
      Absolute Value & \( \left| a \right| \) & \texttt{abs(a)} & \\
      Square Root & \( \sqrt{a} \) & \texttt{sqrt(a)} & \\
      \(n\)th Root & \( \sqrt[n]{a} \) & \texttt{root(a, n)} & \\
      Natural Exponential & \( \exp(a) \) & \texttt{exp(a)} & \\
      Natural Logarithm & \( \ln(a) \) & \texttt{ln(a)} & \\
      Logarithm with Base & \( \log_b{a} \) & \texttt{log(a, b)} & \\[2pt]
      \hline
      Limit & \( \displaystyle\lim_{x\to a} f(x) \) &
        \begin{tabular}{c}
          \texttt{lim(f(x), a, x)} \\
          \texttt{limit(f(x), x, a)}
        \end{tabular} &  \\[2pt]
      Derivative & \( \displaystyle \frac{\mathrm{d}}{\mathrm{d}x} \left( f(x) \right) \) & \texttt{diff(f(x), x)} & \\[8pt]
      Partial Derivative & \( \displaystyle \frac{\partial}{\partial x} \left( f(x,y) \right) \) & \texttt{pdiff(f(x,y), x)} & \\
      Indefinite Integral & \( \displaystyle \int f(x) \,\mathrm{d}x \) & \texttt{int(f(x),x)} & \\[8pt]
      Definite Integral & \( \displaystyle \int_a^b f(x) \,\mathrm{d}x \) & \texttt{int(f(x),x,a,b)} & \\[8pt]
      Sum Over Set & \( \displaystyle \sum_{i \in S} a_i \) & \texttt{sum(a\&\_i, i in S)} & \\[8pt]
      Sum Over Range & \( \displaystyle \sum_{i=a}^b a_i \) & \texttt{sum(a\&\_i,i,a,b)} & \\[8pt]
      Product Over Set & \( \displaystyle \prod_{i \in S} a_i \) &
        \begin{tabular}{c}
          \texttt{prod(a\&\_i, i in S)} \\
          \texttt{product(a\&\_i, i in S)}
        \end{tabular} &  \\[8pt]
      Product Over Range & \( \displaystyle \prod_{i=a}^b a_i \) &
        \begin{tabular}{c}
          \texttt{prod(a\&\_i,a,b)} \\
          \texttt{product(a\&\_i,a,b)}
        \end{tabular} &  \\[8pt]
      \hline
      Universal Quantifier & \( \forall x P(x) \) & \texttt{forall x : \!\!P(x)} & ``For all''; chain nested quantifiers with \texttt{;} \\
      Existential Quantifier & \( \exists x P(x) \) & \texttt{exists x : \!\!P(x)} & ``There exists'' \\
      Unique Quantifier & \( \exists ! x P(x) \) & \texttt{unique x : \!\!P(x)} & ``There exists a unique'' \\
      \hline
    \end{tabular}
  \end{table}
  % subsection functions (end)
  % section syntax (end)

  \newpage

  \section{Symbols by Topic} % (fold)
  \label{sec:topic}
  \subsection{Arithmetic} % (fold)
  \label{sub:arithmetic}
  \begin{table}[!h]
    \centering
    \begin{tabular}{|c|c|c|l|}
      \hline
      \textbf{Name} & \textbf{Symbol} & \textbf{Code} & \textbf{Description} \\
      \hline\hline
      \( i \) & \( i \) & \texttt{i} & \( \sqrt{-1} \)\\
      Plus, Positive & \( + \) & \texttt{+} & binary or unary\\
      Minus, Negative & \( - \) & \texttt{-} & binary or unary\\
      Plus/Minus & \( \pm \) & \texttt{\&pm, +/-} & binary or unary\\
      Minus/Plus & \( \mp \) & \texttt{\&mp, -/+} & binary or unary\\
      Times & \( \cdot \) & \texttt{*} & \\
      Divide & \( / \) & \texttt{/} & \\
      Power & \( a^b \) & \texttt{a\^{}b} & \\
      Square Root & \( \sqrt{a} \) & \texttt{sqrt(a)} & \\
      n-th Root & \( \sqrt[n]{a} \) & \texttt{root(a,n)} & \\
      Log Base n & \( \log_n{a} \) & \texttt{log(a,n)} & \\
      Natural Exponential & \( \exp(a), e^a \) & \texttt{exp(a), e\^{}a} & \\
      Natural Logarithm & \( \ln(a) \) & \texttt{ln(a)} & \\
      Absolute Value & \( \left| a \right| \) & \texttt{abs(a)} & \\
      Factorial & \( ! \) & \texttt{!} & \\
      Modulus & \( a \pmod{n} \) & \texttt{a\%n}, \texttt{a mod n} & \\
      \hline
      Equal & \( = \) & \texttt{=}, \texttt{==} & \\
      Not Equal & \( \ne \) & \texttt{!=}, \texttt{/=}, \texttt{<>} & \\
      Less Than & \( < \) & \texttt{<} & \\
      Greater Than & \( > \) & \texttt{>} & \\
      Less Than or Equal & \( \le \) & \texttt{<=} & \\
      Greater Than or Equal & \( \ge \) & \texttt{>=} & \\
      Parentheses & \( ( \quad ) \) & \texttt{( )} & \\
      \hline
    \end{tabular}
  \end{table}
  % subsection Arithmetic (end)

  \subsection{Algebra} % (fold)
  \label{sub:algebra}
  \begin{table}[!h]
    \centering
    \begin{tabular}{|c|c|c|l|}
      \hline
      \textbf{Name} & \textbf{Symbol} & \textbf{Code} & \textbf{Description} \\
      \hline\hline
      Natural Numbers & \( \mathbb{N} \) & \texttt{\#N} & \\
      Integers & \( \mathbb{Z} \) & \texttt{\#Z} & \\
      Rational Numbers & \( \mathbb{Q} \) & \texttt{\#Q} & \\
      Real Numbers & \( \mathbb{R} \) & \texttt{\#R} & \\
      Complex Numbers & \( \mathbb{C} \) & \texttt{\#C} & \\
      \hline
      Function Composition & \( f \circ g \) & \texttt{f @ g} & \\[2pt]
      \hline
      Sum Over Set & \( \displaystyle \sum_{i \in S} a_i \) & \texttt{sum(a\&\_i, i in S)} & \\[8pt]
      Sum Over Range & \( \displaystyle \sum_{i=a}^b a_i \) & \texttt{sum(a\&\_i,i,a,b)} & \\[4pt]
      Product Over Set & \( \displaystyle \prod_{i \in S} a_i \) &
        \begin{tabular}{c}
          \texttt{prod(a\&\_i, i in S)} \\
          \texttt{product(a\&\_i, i in S)}
        \end{tabular} &  \\[8pt]
      Product Over Range & \( \displaystyle \prod_{i=a}^b a_i \) &
        \begin{tabular}{c}
          \texttt{prod(a\&\_i,a,b)} \\
          \texttt{product(a\&\_i,a,b)}
        \end{tabular} &  \\[8pt]
      \hline
    \end{tabular}
  \end{table}
  % subsection Algebra (end)

  \newpage

  \subsection{Geometry} % (fold)
  \label{sub:geometry}
  \begin{table}[!h]
    \centering
    \begin{tabular}{|c|c|c|l|}
      \hline
      \textbf{Name} & \textbf{Symbol} & \textbf{Code} & \textbf{Description} \\
      \hline\hline
      Pi & \( \pi \) & \texttt{\#pi, \#p} & 3.14\(\cdots\)\\
      Open Interval & \( \left( a,b \right) \) & \texttt{(:a,b:)} & Exclusive Range Delimiters \\
      Closed Interval & \( \left[ a,b \right] \) & \texttt{[:a,b:]} & Inclusive Range Delimiters \\
      Half-Open Intervals & \( \left[ a,b \right) \) & \texttt{[:a,b:)} & Mixed Range Delimiters \\
      \hline
      Vector Components & \( \left\langle a,b,c \right\rangle \) & \texttt{<a,b,c>}, \texttt{<:a,b,c:>} & \\
      Vector & \( \vec{a} \) & \texttt{\&v a} & \\
      Unit Vector & \( \hat{a} \) & \texttt{\&u a} & \\
      Vector Length & \( \left| \vec{a} \right|, \left\| \vec{a} \right\| \) &
        \begin{tabular}{c}
          \texttt{|\&v a|}, \texttt{|:\&v a:|}, \\
          \texttt{||\&v a||}, \texttt{||:\&v a:||}
        \end{tabular} & \\[2pt]
      Zero Vector & \( \vec0 \) & \texttt{\#v0} & \\
      \( x \) Unit Vector & \( \hat\imath \) & \texttt{\#vi} & \\
      \( y \) Unit Vector & \( \hat\jmath \) & \texttt{\#vj} & \\
      \( z \) Unit Vector & \( \hat{k} \) & \texttt{\#vk} & \\
      Dot Product & \( \vec{a} \cdot \vec{b} \) & \texttt{\&v a \&. \&v b} & \\
      Cross Product & \( \vec{a} \times \vec{b} \) & \texttt{\&v a \&x \&v b} & \\
      \hline
    \end{tabular}
  \end{table}
  % subsection Geometry (end)

  \subsection{Trigonometry} % (fold)
  \label{sub:trigonometry}
  \begin{table}[!h]
    \centering
    \begin{tabular}{|c|c|c|l|}
      \hline
      \textbf{Name} & \textbf{Symbol} & \textbf{Code} & \textbf{Description} \\
      \hline\hline
      Trig & \( \sin(\theta) \), \ldots & \texttt{sin(theta)}, \ldots & Also cos, tan, cot, sec, csc \\
      Inverse Trig & \( \arcsin(x) \), \ldots & \texttt{arcsin(x)}, \ldots & Also arccos, arctan, arccot, arcsec, arccsc \\
      Hyperbolic Trig & \( \sinh(\lambda) \), \ldots & \texttt{sinh(lambda)}, \ldots & Also cosh, tanh, coth, sech, csch \\
      Inv. Hyp. Trig & \( \mathrm{arcsinh}(x) \), \ldots & \texttt{arcsinh(x)}, \ldots & Also arccosh, arctanh, arccoth, arcsech, arccsch \\
      \hline
    \end{tabular}
  \end{table}
  % subsection Trigonometry (end)

  \subsection{Discrete} % (fold)
  \label{sub:discrete}
  \begin{table}[!h]
    \centering
    \begin{tabular}{|c|c|c|l|}
      \hline
      \textbf{Name} & \textbf{Symbol} & \textbf{Code} & \textbf{Description} \\
      \hline\hline
      Natural Numbers & \( \mathbb{N} \) & \texttt{\#N} & \\
      Integers & \( \mathbb{Z} \) & \texttt{\#Z} & \\
      \hline
      Factorial & \( ! \) & \texttt{!} & \\
      Modulus & \( a \pmod{n} \) & \texttt{a\%n}, \texttt{a mod n} & \\
      \hline
      Sum Over Set & \( \displaystyle \sum_{i \in S} a_i \) & \texttt{sum(a\&\_i, i in S)} & \\[8pt]
      Sum Over Range & \( \displaystyle \sum_{i=a}^b a_i \) & \texttt{sum(a\&\_i,i,a,b)} & \\[8pt]
      Product Over Set & \( \displaystyle \prod_{i \in S} a_i \) &
        \begin{tabular}{c}
          \texttt{prod(a\&\_i, i in S)} \\
          \texttt{product(a\&\_i, i in S)}
        \end{tabular} &  \\[8pt]
      Product Over Range & \( \displaystyle \prod_{i=a}^b a_i \) &
        \begin{tabular}{c}
          \texttt{prod(a\&\_i,a,b)} \\
          \texttt{product(a\&\_i,a,b)}
        \end{tabular} &  \\[8pt]
      \hline
    \end{tabular}
  \end{table}
  % subsection Discrete (end)

  \newpage

  \subsection{Calculus} % (fold)
  \label{sub:calculus}
  \begin{table}[!h]
    \centering
    \begin{tabular}{|c|c|c|l|}
      \hline
      \textbf{Name} & \textbf{Symbol} & \textbf{Code} & \textbf{Description} \\
      \hline\hline
      Pi & \( \pi \) & \texttt{\#pi, \#p} & 3.14\(\cdots\)\\
      e & \( \mathrm{e} \) & \texttt{\#e} & 2.718\(\cdots\), Natural Base, Euler-Napier number \\
      gamma & \( \gamma \) & \texttt{\#gamma} & 0.577\(\cdots\), Euler-Mascheroni constant\\
      \hline
      Limit & \( \displaystyle\lim_{x\to a} f(x) \) &
        \begin{tabular}{c}
          \texttt{lim(f(x), a, x)} \\
          \texttt{limit(f(x), x, a)}
        \end{tabular} &  \\
      Derivative & \( \displaystyle \frac{\mathrm{d}}{\mathrm{d}x} \left( f(x) \right) \) & \texttt{diff(f(x), x)} & \\[8pt]
      Partial Derivative & \( \displaystyle \frac{\partial}{\partial x} \left( f(x,y) \right) \) & \texttt{pdiff(f(x,y), x)} & \\
      Prime derivative & \( f' \) & \texttt{f'} & Derivative w.r.t. x or first or only variable\\
      Dot derivative & \( \dot{f} \) & \texttt{f.} & Derivative w.r.t. t or second variable\\
      Finite Sum & \( \displaystyle \sum_{i=1}^n f(x_i) \Delta_i  \) & \texttt{sum(f(x\&\_i)*Delta\&\_i,i,1,n)} & \\[8pt]
      Indefinite Integral & \( \displaystyle \int f(x) \,\mathrm{d}x \) & \texttt{int(f(x),x)} & \\[8pt]
      Definite Integral & \( \displaystyle \int_a^b f(x) \,\mathrm{d}x \) & \texttt{int(f(x),x,a,b)} & \\[4pt]
      Differential & \( \mathrm{d}x \) & \texttt{\&d x} & \\
      Partial Differential & \( \partial x \) & \texttt{\&pd x} & \\
      Infinite Series & \( \displaystyle \sum_{i=1}^\infty a_i \) & \texttt{sum(a\&\_i,i,1,infinity)} & \\[4pt]
      \hline
    \end{tabular}
  \end{table}
  % subsection Calculus (end)

  \subsection{Set Theory} % (fold)
  \label{sub:settheory}
  \begin{table}[!h]
    \centering
    \begin{tabular}{|c|c|c|l|}
      \hline
      \textbf{Name} & \textbf{Symbol} & \textbf{Code} & \textbf{Description} \\
      \hline\hline
      Set Delimiters & \( \left\{ \ \right\} \) & \texttt{\{ \}} & \\
      Such That & \( p : q \), \( p ~|~ q \) & \texttt{p : \!\!q}, \texttt{p | q} & \\
      \hline
      Universal Set & \( \mathbb{U} \) & \texttt{\#U} & \\
      Empty Set & \( \emptyset \) & \texttt{\#empty}, \texttt{\{\}} & \\
      \hline
      Natural Numbers & \( \mathbb{N} \) & \texttt{\#N} & \\
      Integer Ring & \( \mathbb{Z} \) & \texttt{\#Z} & \\
      Rational Field & \( \mathbb{Q} \) & \texttt{\#Q} & \\
      Real Field & \( \mathbb{R} \) & \texttt{\#R} & \\
      Complex Field & \( \mathbb{C} \) & \texttt{\#C} & \\
      Quaternion Ring & \( \mathbb{H} \) & \texttt{\#H} & Hamilton numbers\\
      Octonion Algebra & \( \mathbb{O} \) & \texttt{\#O} & Cayley numbers, Type ``Oh''.\\
      \hline
      Subset & \( \subseteq \) & \texttt{subset} & \\
      Superset & \( \supseteq \) & \texttt{superset, supset} & \\
      Proper Subset & \( \subset \) &
        \texttt{propersubset, propsubset, psubset} & \\
      Proper Superset & \( \supset \) &
        \begin{tabular}{c}
          \texttt{propersuperset, propsuperset, psuperset} \\
          \texttt{propersupset, propsupset, psupset}
        \end{tabular} &  \\
      Inclusion & \( \in \) & \texttt{in} & \\
      \hline
      Set Union & \( a \cup b \) & \texttt{a union b} & \\
      Set Intersection & \( a \cap b \) & \texttt{a intersect b} & \\
      Set Difference & \( a \setminus b \) & \texttt{a \textbackslash \ b} & \\
      \hline
    \end{tabular}
  \end{table}
  % subsection Set Theory (end)

  \newpage

  \subsection{Logic} % (fold)
  \label{sub:logic}
  \begin{table}[!h]
    \centering
    \begin{tabular}{|c|c|c|l|}
      \hline
      \textbf{Name} & \textbf{Symbol} & \textbf{Code} & \textbf{Description} \\
      \hline\hline
      True & \( \mathbf{T} \) & \texttt{\#T}, \texttt{\#t}, \texttt{\#true}, \texttt{true} & \\
      False & \( \mathbf{F} \) & \texttt{\#F}, \texttt{\#f}, \texttt{\#false}, \texttt{false} & \\
      \hline
      Conjunction & \( p \wedge q \) & \texttt{p and q}, \texttt{p \&\& q} & Logical AND \\
      Disjunction & \( p \vee q \) & \texttt{p or q}, \texttt{p || q} & Logical OR \\
      Exclusion & \( p \oplus q \) & \texttt{p xor q} & Logical XOR \\
      Logical Negation & \( \neg p \) & \texttt{not p, \~{}p, !p} & \\
      Conditional & \( p \rightarrow q \) & 
        \begin{tabular}{c}
          \texttt{p implies q}, \texttt{p -> q}, \texttt{q if p}, \\
          \texttt{q when p}, \texttt{q whenever p}
        \end{tabular} & \\
      Biconditional & \( p \leftrightarrow q \) & \texttt{p iff q}, \texttt{p <-> q} & \\
      Equivalent & \( \equiv \) & \texttt{===}, \texttt{equiv} & \\
      Not Equivalent & \( \not\equiv \) & \texttt{!==}, \texttt{/==}, \texttt{nequiv} & \\
      \hline
      Universal Quantifier & \( \forall x P(x) \) & \texttt{forall x : \!\!P(x)} & ``For all''; chain nested quantifiers with \texttt{;} \\
      Existential Quantifier & \( \exists x P(x) \) & \texttt{exists x : \!\!P(x)} & ``There exists'' \\
      Unique Quantifier & \( \exists ! x P(x) \) & \texttt{unique x : \!\!P(x)} & ``There exists a unique'' \\
      \hline
    \end{tabular}
  \end{table}
  % subsection Logic (end)

  \subsection{Linear Algebra} % (fold)
  \label{sub:linearalgebra}
  \begin{table}[!h]
    \centering
    \begin{tabular}{|c|c|c|l|}
      \hline
      \textbf{Name} & \textbf{Symbol} & \textbf{Code} & \textbf{Description} \\
      \hline\hline
      Vector Delimiters & \( \left\langle \ \right\rangle \) & \texttt{< >}, \texttt{<: :>} & \\[2 pt]
      Zero Vector & \( \vec0 \) & \texttt{\#v0} & \\
      \( x \) Unit Vector & \( \hat\imath \) & \texttt{\#vi} & \\
      \( y \) Unit Vector & \( \hat\jmath \) & \texttt{\#vj} & \\
      \( z \) Unit Vector & \( \hat{k} \) & \texttt{\#vk} & \\
      \hline
      Matrix Delimiters & \( \left[ \ \right] \) & \texttt{[ ]} & \\
      Zero Matrix & \( \mathbb{O} \) & \texttt{\#0} & Type ``zero''.\\
      Unit Matrix & \( \mathbb{I} \) & \texttt{\#1} & Type ``one''.\\
      \hline
    \end{tabular}
  \end{table}
  % subsection Linear Algebra (end)

  % section topics (end)

\end{document}
